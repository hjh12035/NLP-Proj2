% This must be in the first 5 lines to tell arXiv to use pdfLaTeX, which is strongly recommended.
\pdfoutput=1
% In particular, the hyperref package requires pdfLaTeX in order to break URLs across lines.


\documentclass[11pt]{article}

% Remove the "review" option to generate the final version.
\usepackage{ACL2023}
\usepackage{CJK}
% Standard package includes
\usepackage{times}
\usepackage{latexsym}
% \usepackage[UTF8]{ctex}

\usepackage{amsmath}

% For proper rendering and hyphenation of words containing Latin characters (including in bib files)
\usepackage[T1]{fontenc}
% For Vietnamese characters
% \usepackage[T5]{fontenc}
% See https://www.latex-project.org/help/documentation/encguide.pdf for other character sets

% This assumes your files are encoded as UTF8
\usepackage[utf8]{inputenc}

% This is not strictly necessary, and may be commented out.
% However, it will improve the layout of the manuscript,
% and will typically save some space.
\usepackage{microtype}
\renewcommand{\tablename}{表格}
% This is also not strictly necessary, and may be commented out.
% However, it will improve the aesthetics of text in
% the typewriter font.
% \usepackage{inconsolata}


% If the title and author information does not fit in the area allocated, uncomment the following
%
%\setlength\titlebox{<dim>}
%
% and set <dim> to something 5cm or larger.



% Author information can be set in various styles:
% For several authors from the same institution:
% \author{Author 1 \and ... \and Author n \\
%         Address line \\ ... \\ Address line}
% if the names do not fit well on one line use
%         Author 1 \\ {\bf Author 2} \\ ... \\ {\bf Author n} \\
% For authors from different institutions:
% \author{Author 1 \\ Address line \\  ... \\ Address line
%         \And  ... \And
%         Author n \\ Address line \\ ... \\ Address line}
% To start a seperate ``row'' of authors use \AND, as in
% \author{Author 1 \\ Address line \\  ... \\ Address line
%         \AND
%         Author 2 \\ Address line \\ ... \\ Address line \And
%         Author 3 \\ Address line \\ ... \\ Address line}



\begin{document}
\begin{CJK}{UTF8}{gbsn}

% TODO: 从这里开始写

\title{基于检索增强生成的智能课程助教系统}

\author{胡健豪 \\
  523031910287 \\\And
  姚明哲 \\
  523031910408 \\}

\maketitle

\begin{abstract}
本项目实现了一个基于检索增强生成(Retrieval-Augmented Generation, RAG)技术的智能课程助教系统。针对大语言模型(LLM)在特定领域知识上的幻觉问题和时效性限制,本系统通过构建课程文档的向量知识库,实现了基于语义检索的精准问答。系统支持PDF、PPTX、DOCX等多种格式文档的解析与切分,利用ChromaDB进行向量存储,并结合OpenAI API进行回答生成。此外,本项目还扩展了自动习题生成和课程复习提纲生成功能,并通过React+FastAPI构建了友好的Web交互界面。实验结果表明,该系统能有效缓解LLM的幻觉问题,提升回答的专业性和可信度。
\end{abstract}

\section{引言}

随着大语言模型(LLM)的发展,AI在教育领域的应用日益广泛。然而,直接使用通用LLM作为课程助教存在明显局限:一是模型训练数据存在时间截止,无法获取最新的课程调整信息;二是模型容易产生“幻觉”,生成看似合理但与课程内容不符的错误答案;三是受限于上下文窗口,难以一次性处理大量的教材和课件。

为了解决上述问题,本项目采用了检索增强生成(RAG)技术。RAG通过在生成回答前检索外部知识库中的相关信息,将检索到的准确片段作为上下文输入给LLM,从而确保回答的准确性和可验证性。本项目旨在搭建一个智能课程助教系统,不仅能回答学生关于课程内容的提问,还能辅助生成练习题和复习提纲,为学生提供全方位的学习支持。

\section{系统设计}

本系统的整体架构包含数据处理流水线、向量检索模块、RAG核心代理以及用户交互界面四个部分。

\subsection{文档处理模块}
文档处理是RAG系统的基础。我们实现了\texttt{DocumentLoader}类,支持多种格式文档的加载:
\begin{itemize}
    \item \textbf{PDF处理}:使用\texttt{PyPDF2}库按页提取文本。
    \item \textbf{PPTX处理}:使用\texttt{python-pptx}库遍历幻灯片提取文本框内容。
    \item \textbf{DOCX/TXT处理}:分别使用\texttt{docx2txt}和原生文件读取方式提取文本。
\end{itemize}

为了适应Embedding模型的输入限制并保留上下文,我们实现了\texttt{TextSplitter}类。采用滑动窗口策略,将长文本切分为固定长度(如500字符)的块(Chunk),并设置重叠区域(Overlap,如50字符),同时优化了切分逻辑,尽量在句子结束符(如句号、问号)处截断,以保证语义的完整性。

\subsection{向量数据库模块}
向量存储与检索模块基于\texttt{ChromaDB}实现。
\begin{itemize}
    \item \textbf{Embedding}:调用OpenAI的\texttt{text-embedding-3-small}模型将文本块转化为高维向量。
    \item \textbf{存储}:将文本向量及其元数据(文件名、页码、块ID)存入ChromaDB集合中。
    \item \textbf{检索}:对于用户Query,同样生成向量,计算其与库中向量的余弦相似度,返回Top-K个最相关的文档片段。
\end{itemize}

\subsection{RAG智能体 (RAG Agent)}
\texttt{RAGAgent}是系统的核心大脑,主要负责:
\begin{enumerate}
    \item \textbf{检索上下文}:调用向量数据库接口获取相关资料。
    \item \textbf{构建Prompt}:设计了专门的System Prompt,确立助教“专业、友好、苏格拉底式引导”的人设。在User Prompt中,将检索到的上下文片段以“---课程资料---”的形式注入,并要求模型基于资料回答,若资料不足则明确告知。
    \item \textbf{生成回答}:调用GPT-4o-mini(或同类模型)生成最终回复,并附带来源标注(如“来源:lecture1.pdf 第5页”)。
\end{enumerate}

\subsection{功能扩展}
除了基础问答,我们还实现了以下扩展功能:
\begin{itemize}
    \item \textbf{自动习题生成}:设计了特定的Prompt,要求模型根据检索到的考点,输出JSON格式的习题(包含题干、选项、答案、解析),支持选择题和简答题。
    \item \textbf{复习提纲生成}:实现了递归生成的Prompt策略,能够为指定主题或全课程生成层级化的复习大纲,并在前端以树形结构展示。
    \item \textbf{Web交互界面}:前端采用Next.js + Tailwind CSS,后端使用FastAPI,实现了类似ChatGPT的流式对话体验,以及专门的习题和提纲展示页面。
\end{itemize}

\section{实验与结果展示}

为了验证系统的有效性,我们使用了[课程名称,如NLP]课程的课件作为知识库进行了测试。

\subsection{基础问答测试}
\textbf{问题1:} [请在此处填入一个测试问题,例如:什么是Transformer?] \\
\textbf{系统回答:} [请在此处填入系统的实际回答] \\
\textbf{分析:} [请简要分析回答是否准确,是否引用了正确的文档]

\textbf{问题2:} [请在此处填入另一个测试问题] \\
\textbf{系统回答:} [请在此处填入系统的实际回答] \\
\textbf{分析:} ...

\subsection{习题生成测试}
\textbf{输入主题:} [例如:注意力机制] \\
\textbf{生成结果:} [请截图或复制生成的题目]

\subsection{复习提纲测试}
\textbf{输入主题:} [例如:全课程] \\
\textbf{生成结果:} [请截图或复制生成的提纲结构]

\section{总结与讨论}

本项目成功构建了一个功能完善的RAG课程助教系统。通过引入外部知识库,系统有效解决了LLM在特定课程内容上的幻觉问题,能够提供可追溯的准确回答。扩展的习题与提纲功能进一步丰富了系统的应用场景。

\textbf{局限性与展望:}
\begin{itemize}
    \item 当前主要处理文本信息,对于课件中的图表、公式(尤其是图片格式的公式)理解能力有限。未来可引入多模态大模型(如GPT-4o)增强对非文本内容的解析。
    \item 检索策略目前主要依赖语义相似度,对于关键词匹配要求高的场景可能不够精确。未来可尝试混合检索(Hybrid Search)策略,结合BM25算法。
\end{itemize}

\section{成员分工}

\begin{itemize}
    \item \textbf{姓名1}:负责...(例如:文档处理模块与向量数据库搭建,实验报告撰写)
    \item \textbf{姓名2}:负责...(例如:RAG Agent核心逻辑实现,Prompt设计与优化)
    \item \textbf{姓名3}:负责...(例如:Web前端与后端API开发,系统集成与测试)
\end{itemize}

% \bibliography{anthology,custom}
% \bibliographystyle{acl_natbib}

\end{CJK}
\end{document}

% This must be in the first 5 lines to tell arXiv to use pdfLaTeX, which is strongly recommended.
\pdfoutput=1
% In particular, the hyperref package requires pdfLaTeX in order to break URLs across lines.


\documentclass[11pt]{article}

% Remove the "review" option to generate the final version.
\usepackage{ACL2023}
\usepackage{CJK}
% Standard package includes
\usepackage{times}
\usepackage{latexsym}
% \usepackage[UTF8]{ctex}

\usepackage{amsmath}

% For proper rendering and hyphenation of words containing Latin characters (including in bib files)
\usepackage[T1]{fontenc}
% For Vietnamese characters
% \usepackage[T5]{fontenc}
% See https://www.latex-project.org/help/documentation/encguide.pdf for other character sets

% This assumes your files are encoded as UTF8
\usepackage[utf8]{inputenc}

% This is not strictly necessary, and may be commented out.
% However, it will improve the layout of the manuscript,
% and will typically save some space.
\usepackage{microtype}
\renewcommand{\tablename}{表格}
% This is also not strictly necessary, and may be commented out.
% However, it will improve the aesthetics of text in
% the typewriter font.
% \usepackage{inconsolata}


% If the title and author information does not fit in the area allocated, uncomment the following
%
%\setlength\titlebox{<dim>}
%
% and set <dim> to something 5cm or larger.



% Author information can be set in various styles:
% For several authors from the same institution:
% \author{Author 1 \and ... \and Author n \\
%         Address line \\ ... \\ Address line}
% if the names do not fit well on one line use
%         Author 1 \\ {\bf Author 2} \\ ... \\ {\bf Author n} \\
% For authors from different institutions:
% \author{Author 1 \\ Address line \\  ... \\ Address line
%         \And  ... \And
%         Author n \\ Address line \\ ... \\ Address line}
% To start a seperate ``row'' of authors use \AND, as in
% \author{Author 1 \\ Address line \\  ... \\ Address line
%         \AND
%         Author 2 \\ Address line \\ ... \\ Address line \And
%         Author 3 \\ Address line \\ ... \\ Address line}



\begin{document}
\begin{CJK}{UTF8}{gbsn}

% TODO: 从这里开始写

\title{基于检索增强生成的智能课程助教系统}

\author{胡健豪 \\
  523031910287 \\\And
  姚明哲 \\
  523031910408 \\}

\maketitle

\begin{abstract}
本项目实现了一个基于检索增强生成(Retrieval-Augmented Generation, RAG)技术的智能课程助教系统。针对大语言模型(LLM)在特定领域知识上的幻觉问题和时效性限制,本系统通过构建课程文档的向量知识库,实现了基于语义检索的精准问答。系统支持PDF、PPTX、DOCX等多种格式文档的解析与切分,利用ChromaDB进行向量存储,并结合OpenAI API进行回答生成。此外,本项目还扩展了自动习题生成和课程复习提纲生成功能,并通过React+FastAPI构建了友好的Web交互界面。实验结果表明,该系统能有效缓解LLM的幻觉问题,提升回答的专业性和可信度。
\end{abstract}

\section{引言}

随着大语言模型(LLM)的发展,AI在教育领域的应用日益广泛。然而,直接使用通用LLM作为课程助教存在明显局限:一是模型训练数据存在时间截止,无法获取最新的课程调整信息;二是模型容易产生“幻觉”,生成看似合理但与课程内容不符的错误答案;三是受限于上下文窗口,难以一次性处理大量的教材和课件。

为了解决上述问题,本项目采用了检索增强生成(RAG)技术。RAG通过在生成回答前检索外部知识库中的相关信息,将检索到的准确片段作为上下文输入给LLM,从而确保回答的准确性和可验证性。本项目旨在搭建一个智能课程助教系统,不仅能回答学生关于课程内容的提问,还能辅助生成练习题和复习提纲,为学生提供全方位的学习支持。

\section{系统设计}
整体上,系统采用“离线预处理 + 在线检索生成”的架构。
离线阶段通过 \texttt{process\_data.py} 遍历课程数据目录,对文档进行加载、切分与向量化,并写入本地 ChromaDB;
在线阶段由 RAG Agent 接收用户问题,调用向量库检索上下文,并据此构造提示词、调用 LLM 生成回答。
下面分别介绍各核心模块。

\subsection{文档处理模块}
文档处理模块由 \texttt{DocumentLoader} 与 \texttt{TextSplitter} 组成,负责从多种格式的课程资料中抽取可检索的文本块,并附带必要的位置信息(如文件名与页码)。

在 PDF 与 PPTX 场景下,我们将“页/幻灯片”作为自然的切分单位。
对于 PDF,\texttt{load\_pdf} 使用 \texttt{PdfReader} 逐页提取文本,并统一封装为形如“\verb|--- 第 X 页 ---|”的前缀结构;
对于 PPTX,\texttt{load\_pptx} 通过 \texttt{Presentation} 遍历每一张幻灯片,
对所有具备 \texttt{text} 属性的形状进行聚合,并以“\verb|--- 幻灯片 X ---|”标记。
这样的设计一方面方便之后在回答中精确引用“第几页/第几张幻灯片”,另一方面也为后续的图片 OCR 与文本块融合提供位置对齐依据。

对于 DOCX 与 TXT 文件,我们认为其天然是线性长文本,不适合简单按页/行切分。
因而 \texttt{load\_docx} 与 \texttt{load\_txt} 仅负责读取完整字符串,实际的块划分则交由 \texttt{TextSplitter} 完成。
该类在初始化时接收 \texttt{chunk\_size} 与 \texttt{chunk\_overlap} 两个超参数,
分别控制每个文本块的最大长度与相邻块之间的重叠长度。

在 \texttt{split\_text} 中,我们采取“窗口 + 句边界优先”的策略:
从当前起点 \texttt{start} 出发,先在 \texttt{[start, start + chunk\_size]} 范围内确定候选片段,
再从后往前扫描句子结束符(中文的“。!?”、英文的 \verb!.!?! 以及换行符),
若能在窗口内找到合适的句末位置,则优先在此处分割;若找不到,则退化为按定长切分。
之后,新块的起点被设定为 \texttt{end - chunk\_overlap},
通过加入“如果没有前进至少 1 个字符则强制前进”的保护,避免在极端情况下出现死循环。

这一策略在实践中取得了两方面收益:
一是块长度在模型上下文限制之内,同时保留了足够的语义信息;
二是块之间的重叠使得跨句甚至跨段的逻辑关联得以保留,从而提高了检索结果在回答生成中的可用性。

此外,为了利用课件中的图片信息(如截图公式、流程图和板书),
\texttt{DocumentLoader} 还实现了 \texttt{extract\_images\_from\_pdf} 和 \texttt{extract\_images\_from\_pptx}。
这两个函数通过 \texttt{fitz} 与 PPTX 的图形接口枚举所有图片对象,
利用最小边长阈值过滤掉图标和装饰性小图,将剩余图片保存到 \verb|images| 目录,
然后调用 \texttt{pytesseract} 进行 OCR 识别。
我们只保留字符数和词数均超过预设阈值的 OCR 文本,以降低噪声。
最终,图片块也被封装为带有页码与图片 ID 的文档单元,与纯文本一同进入下游向量化流程。

\subsection{向量数据库模块}
向量数据库模块由 \texttt{VectorStore} 实现,其职责是将预处理后的文档块映射为向量表示,并支持基于语义相似度的检索。
我们选择 Chroma 作为本地持久化向量库,一方面是因为其 Python 接口简洁,
另一方面是其对元数据与 ID 管理的支持较好,便于后续做来源追踪。

在 \texttt{add\_documents} 中,系统对每个文档块提取 \texttt{content} 字段,
调用 \texttt{get\_embedding} 使用 OpenAI 提供的 Embedding 模型生成稠密向量。
考虑到换行符可能干扰 embedding 质量,我们在请求前将文本中的换行统一替换为空格。
每个块被分配一个唯一 ID:对于文本块使用“\verb|filename_p{page}_c{chunk}|”,
对于图片块使用“\verb|filename_p{page}_img{image\_id}|”,
对应的元数据中保留了文件名、文件路径、文件类型和页码等信息。
所有向量、ID、文本与元数据最终批量写入 Chroma collection 中。

为了提高检索的鲁棒性,我们在同一模块中实现了 BM25 稀疏检索。
具体而言,\texttt{\_tokenize} 将文本转为简单的英语单词序列(对于中文可以扩展为更合理的分词方案),
所有块的 token 列表被用于构建 \texttt{BM25Okapi} 模型。
向量检索通过 \texttt{search} 完成,BM25 则由 \texttt{bm25\_search} 提供。
在此基础上,我们实现了一个基于 Reciprocal Rank Fusion (RRF) 的混合检索接口 \texttt{hybrid\_search}:
分别对稠密与稀疏检索结果按排名位置计算倒数得分,并对相同文档 ID 做加和排序。
尽管当前在线 RAG Agent 主要使用向量检索,但该混合策略在实验中针对短问句与术语查询表现出更好的稳定性,为后续进一步集成打下了基础。

\texttt{process\_data.py} 将上述流程串联起来:
检查数据目录存在性,调用 \texttt{DocumentLoader.load\_all\_documents} 递归加载所有支持格式的文件,
将得到的文档列表送入 \texttt{TextSplitter.split\_documents} 完成二次切分,
最后通过 \texttt{VectorStore.add\_documents} 完成整体向量索引构建。
在每次运行前,系统会清空现有 collection,以便在课程资料更新后重新构建。

\subsection{RAG智能体 (RAG Agent)}
在线对话逻辑由 \texttt{RAGAgent} 统一管理。
该类封装了与 LLM 的交互方式、检索策略以及对话级别的状态维护。

首先,我们通过 \texttt{system\_prompt} 明确约束模型的“人格”和行为边界:
系统被设定为一名课程助教,需要在回答时尽量引用知识库中提供的课程资料,并在答案中说明信息来源(包括文档名与页码);
对于不同类型的问题(概念问答、作业/练习题、代码与实践问题),助教应采用不同策略,
其中作业题要侧重思路和方法提示,而避免直接给出完整解答;
在交流风格上,则要求其使用中文、保持专业且友好,并适当使用苏格拉底式提问引导学生思考。

在具体问答流程中,\texttt{answer\_question} 首先将用户输入与对话历史传入 \texttt{analyze\_intent}。
我们使用一个轻量模型根据最近几轮对话分析当前问题的意图,
输出诸如 \texttt{NEW\_TOPIC}、\texttt{DRILL\_DOWN}、\texttt{TOPIC\_SHIFT}、
\texttt{CLARIFICATION}、\texttt{SUMMARIZATION} 或 \texttt{CHIT\_CHAT} 等类别,
以及一个经过消歧与补全指代的 \texttt{rewritten\_query}。
意图分析的目的有二:一是将“这个”“它”等指代词还原为具体概念,从而提升检索准确性;
二是为后续的上下文窗口管理提供依据。

\texttt{RAGAgent} 维护了一个有限长度的上下文窗口 \texttt{context\_window},该列表以最新检索到的文档片段为主,
当意图为 \texttt{NEW\_TOPIC} 或强烈纠正类时,窗口会被部分或全部重置;
当意图为 \texttt{DRILL\_DOWN} 或 \texttt{TOPIC\_SHIFT} 时,
则在原有窗口基础上追加新检索结果,并通过简单策略控制窗口长度与去重。
这样一来,系统在多轮对话中既能保持主题连续,又能避免无休止地扩大上下文开销。

在检索阶段,\texttt{retrieve\_context} 调用 \texttt{VectorStore} 根据重写后的查询获取若干文档片段,
并通过 \texttt{\_format\_context} 将它们拼接为适合填入 Prompt 的结构化文本,
其中显式包含文件名与页码,便于模型在回答中进行引用。
最终,\texttt{generate\_response} 将系统提示词、若干轮对话历史(可选)
以及格式化后的“课程资料 + 学生问题”作为消息列表,调用 OpenAI Chat Completion 模型生成答案。
对于需要流式输出的场景,接口也预留了增量返回机制,以适配 Web 前端的实时对话体验。

\subsection{功能扩展}
在上述基础问答功能之外,我们还实现了若干与教学场景高度相关的扩展功能。

首先,在自动习题生成方面,系统提供了 \texttt{generate\_quiz} 接口。
用户指定主题(如“注意力机制”)、难度等级和题型(选择题或简答题),
系统会通过 \texttt{\_expand\_query} 使用小模型将该主题扩展为
若干更具体的检索关键词(例如加入“定义”“优点”“典型公式”等),
然后利用向量库检索覆盖面更广的上下文。
随后,我们为每一道题构造一个专用的 Prompt,要求模型在给定资料范围内生成高质量题目,
并严格按照预先约定的 JSON 格式输出题干、选项、参考答案、解析与来源信息。
题目的生成过程使用多线程并行,以加速在前端一次性展示多题的体验。

其次,在复习提纲方面,\texttt{generate\_outline} 支持针对“全课程”或任意子主题生成分层的 Markdown 提纲。
与习题生成类似,我们先通过查询扩展与检索获取与主题高度相关的多个文档片段,
再通过一个强调“层级结构、逻辑清晰与重点突出”的系统提示词,
引导模型产出使用 \#、\#\#、\#\#\# 标记的结构化内容。
前端可以基于这些 Markdown 自动渲染出树形知识结构,帮助学生从宏观上把握课程体系。

最后,为了降低使用门槛,我们基于 React(Next.js)+ Tailwind CSS 实现了简单的 Web 前端,
并用 FastAPI 封装后端接口,实现了三类入口:
对话式课程助教、自动习题生成与复习提纲浏览。
对话界面采用类 ChatGPT 的流式输出方式;
习题与提纲页面则以卡片与树视图的形式展示模型输出,并保留了对应的来源信息。


\section{实验与结果展示}

为了验证系统的有效性,我们使用了[课程名称,如NLP]课程的课件作为知识库进行了测试。

\subsection{基础问答测试}
\textbf{问题1:} [请在此处填入一个测试问题,例如:什么是Transformer?] \\
\textbf{系统回答:} [请在此处填入系统的实际回答] \\
\textbf{分析:} [请简要分析回答是否准确,是否引用了正确的文档]

\textbf{问题2:} [请在此处填入另一个测试问题] \\
\textbf{系统回答:} [请在此处填入系统的实际回答] \\
\textbf{分析:} ...

\subsection{习题生成测试}
\textbf{输入主题:} [例如:注意力机制] \\
\textbf{生成结果:} [请截图或复制生成的题目]

\subsection{复习提纲测试}
\textbf{输入主题:} [例如:全课程] \\
\textbf{生成结果:} [请截图或复制生成的提纲结构]

\section{总结与讨论}

本项目成功构建了一个功能完善的RAG课程助教系统。通过引入外部知识库,系统有效解决了LLM在特定课程内容上的幻觉问题,能够提供可追溯的准确回答。扩展的习题与提纲功能进一步丰富了系统的应用场景。

\textbf{局限性与展望:}
\begin{itemize}
    \item \textbf{多模态理解有限。} 虽然我们通过 OCR 将部分重要图片转化为文本进入知识库,但对于包含复杂公式、图表或混合排版的课件,OCR 易引入噪声,且丢失布局信息。未来可以考虑引入更强的多模态模型(如具备文档理解能力的视觉语言模型),直接处理 PDF 或截图,以提升对图像信息的利用率。
    \item \textbf{检索与生成尚未深度联动。} 目前生成模型主要依赖检索到的静态上下文,对于检索结果的再筛选与聚合仍较为粗糙。后续可以考虑引入基于重排序模型的二次筛选,或在生成过程中动态查询(例如基于 Tool-Calling 的 RAG),以减少无关上下文的干扰。
    \item \textbf{自动评测缺乏。} 本项目主要通过案例分析来评估问答质量和习题/提纲实用性,缺乏统一的量化评测指标。未来可以构建一小规模的“标准问答对”与“目标提纲”,通过 BLEU、ROUGE 或人工打分等方式更系统地比较不同检索/Prompt 策略。
    \item \textbf{部署与成本。} 当前系统依赖云端 LLM 与 Embedding 服务,在大量并发访问或大规模习题生成场景下成本较高。可以考虑引入本地开源模型(如量化后的 LLM 与向量模型),或对索引与缓存策略进行优化,以在性能与成本之间取得更好平衡。
\end{itemize}

\section{成员分工}

\begin{itemize}
    \item \textbf{姓名1}:负责...(例如:文档处理模块与向量数据库搭建,实验报告撰写)
    \item \textbf{姓名2}:负责...(例如:RAG Agent核心逻辑实现,Prompt设计与优化)
    \item \textbf{姓名3}:负责...(例如:Web前端与后端API开发,系统集成与测试)
\end{itemize}

% \bibliography{anthology,custom}
% \bibliographystyle{acl_natbib}

\end{CJK}
\end{document}
